This section describes how the IP core can be configured by means of
Table~\ref{tab:confs}. The core may be configured using parameters or
macros. The parameters are passed to each instance of the core and only affect
that instance. The macros apply to all instances of the core. The macros and
parameters have the following types:
\begin{description}
\item \textbf{'M'} Valued Macro: the value of the macro dictates the way the core is built.
\item \textbf{'P'} True Parameter: the value of the parameter influences how the core instance that receives is built.
\item \textbf{'F'} False Parameter: the parameter needs to be on the parameter list, but cannot be changed by the user.
\end{description}

\begin{xltabular}{\textwidth}{|l|c|c|c|c|X|} \hline
    \rowcolor{iob-green}
    {\bf Macro} & {\bf Type} & {\bf Min} & {\bf Typical} & {\bf Max} & {\bf Description}
    \\ \hline \hline
    \input confs_tab
    \caption{Configuration Macros.}\label{tab:confs}
\end{xltabular}

Note: The 16-bit auto-created Product Version macro uses nibbles to represent
decimal num- bers using their binary values.  The two most significant nib- bles
represent the integral part of the version, and the two least signifi- cant
nibbles represent the decimal part.  For example V12.34 is rep- resented by
0x1234.
