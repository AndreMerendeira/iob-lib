The Quartus and Vivado FPGA compilation toolchains are supported via {\tt .tcl}
scripts invoked by a Makefile. The script compiles and elaborates the design for
a given set of target FPGA boards. Additional boards can be supported by
following the flow provided.

The core is instantiated within a Verilog wrapper, which depends on the
specifics of the FPGA device and the FPGA board. Timing constraints file(s) are
provided.

After compilation, reports on FPGA resource usage, power consumption, and timing
closure are generated. A post-synthesis Verilog file is created, which can be
used in post-synthesis simulation.

Add here any specific ip core description.


\subsubsection*{System-level FPGA Run}

Upon request, files to run the core on the supported FPGA boards can be
provided. The core is embedded in a RISC-V system and exercised in various
modes, using a bare-metal software program written in the C programming
language.
