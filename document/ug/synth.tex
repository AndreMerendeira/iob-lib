\subsection{Synthesis Macros and Parameters}
\label{sec:smp}

\ifdefined\SMP
The synthesis macros and parameters of the core are presented in this section.
For each macro or parameter, the minimum, default and maximum values are
provided. The minimum and maximum values may be marked with

\begin{description}
\item{Not Applicable(NA)}:  the macro or parameter value cannot be set by the user; its value is set automatically by the system.
\item{Not Specified (NS)}:  the minimum or maximum value for the macro or parameter has not been determined; if needed, values other than the default should be determined and tested by the user. 
\end{description}


\ifdefined\SM

The synthesis macros apply to all instances of the core, and are listed in
Table~\ref{tab:sm}.

\begin{table}[h]
  \centering
    \begin{tabularx}{\textwidth}{ | c | c | c | c | X | }
    \hline
    \rowcolor{iob-green}
    {\bf Parameter} & {\bf Min} & {\bf Typ} & {\bf Max} & {\bf Description} \\\hline

    \input{sm_tab}

    \end{tabularx}
  \caption{Synthesis Macros.}
  \label{tab:sm}
\end{table}
\fi


\ifdefined\SP

The generic synthesis parameters of the core are presented in
Table~\ref{tab:sp}. Generic parameters can vary from instance to instance.

\begin{table}[h]
  \centering
    \begin{tabularx}{\textwidth}{ | c | c | c | c | X | }
    \hline
    \rowcolor{iob-green}
    {\bf Parameter} & {\bf Min} & {\bf Typ} & {\bf Max} & {\bf Description} \\\hline

    \input{sp_tab}

    \end{tabularx}
  \caption{Synthesis Parameters.}
  \label{tab:sp}
\end{table}
\fi
\fi


\subsection{Synthesis Script and Timing Constraints}

A simple {\tt .tcl} script is provided for the Cadence Genus synthesis tool. The
script reads the technology files, compiles and elaborates the design, and
proceeds to synthesise it. The timing constraints are contained within the
constraints file provided, or provided in a separate file.

After synthesis, reports on silicon area usage, power consumption, and timing
closure are generated. A post-synthesis Verilog file is created, to be used in
post-synthesis simulation.

\subsection{Synthesis Macros}

\ifdefined\SM

The synthesis macros apply to all instances of the core, and are listed in
Table~\ref{tab:sm}.

\begin{table}[h]
  \centering
    \begin{tabularx}{\textwidth}{ | c | c | c | c | X | }
    \hline
    \rowcolor{iob-green}
    {\bf Parameter} & {\bf Min} & {\bf Typ} & {\bf Max} & {\bf Description} \\\hline

    \input{sm_tab}

    \end{tabularx}
  \caption{Synthesis Macros.}
  \label{tab:sm}
\end{table}

\noindent

\else
The IP core has no user-definable macros other than the default synthesis parameters values, if available.
\fi


\subsection{Synthesis Parameters}

\ifdefined\SP

The generic synthesis parameters of the core are presented in
Table~\ref{tab:sp}. Generic parameters can vary from instance to instance.

\begin{table}[h]
  \centering
    \begin{tabularx}{\textwidth}{ | c | c | c | c | X | }
    \hline
    \rowcolor{iob-green}
    {\bf Parameter} & {\bf Min} & {\bf Typ} & {\bf Max} & {\bf Description} \\\hline

    \input{sp_tab}

    \end{tabularx}
    
\caption{Synthesis Parameters.}
  \label{tab:sp}
\end{table}

\else

This IP core has no synthesis parameters.

\fi


\subsection{Synthesis Script and Timing Constraints}

A simple {\tt .tcl} script is provided for the Cadence Genus synthesis tool. The
script reads the technology files, compiles and elaborates the design, and
proceeds to synthesise it. The timing constraints are contained within the
constraints file provided, or provided in a separate file.

After synthesis, reports on silicon area usage, power consumption, and timing
closure are generated. A post-synthesis Verilog file is created, to be used in
post-synthesis simulation.

\subsection{Synthesis Macros}

\ifdefined\SM

The synthesis macros apply to all instances of the core, and are listed in
Table~\ref{tab:sm}.

\begin{table}[h]
  \centering
    \begin{tabularx}{\textwidth}{ | c | c | c | c | X | }
    \hline
    \rowcolor{iob-green}
    {\bf Parameter} & {\bf Min} & {\bf Typ} & {\bf Max} & {\bf Description} \\\hline

    \input{sm_tab}

    \end{tabularx}
  \caption{Synthesis Macros.}
  \label{tab:sm}
\end{table}

\noindent

\else
The IP core has no user-definable macros other than the default synthesis parameters values, if available.
\fi


\subsection{Synthesis Parameters}

\ifdefined\SP

The generic synthesis parameters of the core are presented in
Table~\ref{tab:sp}. Generic parameters can vary from instance to instance.

\begin{table}[h]
  \centering
    \begin{tabularx}{\textwidth}{ | c | c | c | c | X | }
    \hline
    \rowcolor{iob-green}
    {\bf Parameter} & {\bf Min} & {\bf Typ} & {\bf Max} & {\bf Description} \\\hline

    \input{sp_tab}

    \end{tabularx}
    
\caption{Synthesis Parameters.}
  \label{tab:sp}
\end{table}

\else

This IP core has no synthesis parameters.

\fi


\subsection{Synthesis Script and Timing Constraints}

A simple {\tt .tcl} script is provided for the Cadence Genus synthesis tool. The
script reads the technology files, compiles and elaborates the design, and
proceeds to synthesise it. The timing constraints are contained within the
constraints file provided, or provided in a separate file.

After synthesis, reports on silicon area usage, power consumption, and timing
closure are generated. A post-synthesis Verilog file is created, to be used in
post-synthesis simulation.

\subsection{Synthesis Macros}

\ifdefined\SM

The synthesis macros apply to all instances of the core, and are listed in
Table~\ref{tab:sm}.

\begin{table}[h]
  \centering
    \begin{tabularx}{\textwidth}{ | c | c | c | c | X | }
    \hline
    \rowcolor{iob-green}
    {\bf Parameter} & {\bf Min} & {\bf Typ} & {\bf Max} & {\bf Description} \\\hline

    \input{sm_tab}

    \end{tabularx}
  \caption{Synthesis Macros.}
  \label{tab:sm}
\end{table}

\noindent

\else
The IP core has no user-definable macros other than the default synthesis parameters values, if available.
\fi


\subsection{Synthesis Parameters}

\ifdefined\SP

The generic synthesis parameters of the core are presented in
Table~\ref{tab:sp}. Generic parameters can vary from instance to instance.

\begin{table}[h]
  \centering
    \begin{tabularx}{\textwidth}{ | c | c | c | c | X | }
    \hline
    \rowcolor{iob-green}
    {\bf Parameter} & {\bf Min} & {\bf Typ} & {\bf Max} & {\bf Description} \\\hline

    \input{sp_tab}

    \end{tabularx}
    
\caption{Synthesis Parameters.}
  \label{tab:sp}
\end{table}

\else

This IP core has no synthesis parameters.

\fi


\subsection{Synthesis Script and Timing Constraints}

A simple {\tt .tcl} script is provided for the Cadence Genus synthesis tool. The
script reads the technology files, compiles and elaborates the design, and
proceeds to synthesise it. The timing constraints are contained within the
constraints file provided, or provided in a separate file.

After synthesis, reports on silicon area usage, power consumption, and timing
closure are generated. A post-synthesis Verilog file is created, to be used in
post-synthesis simulation.

\input{synth}



